\documentclass{article}
\usepackage[dvipsnames]{xcolor}
\usepackage{amssymb}

\title{STACS Constitution}
\vspace{-8ex}
\date{}
\begin{document}

\maketitle


\section{Title}

The society shall be called ”University of St Andrews Computing Society”, but
may also be abbreviated to ”STACS”.

\section{Aims}

The aims of the society are:

\begin{itemize}
    \item To organise social and recreational events for our members.
    \item To organise and publicise lectures, discussion forums, and events relating to computing, technology, and computer science.
    \item To organise workshops of interest to those wishing to develop their computing skills, and to expand on the curriculum taught by the university.
    \item To encourage, by the process of engagement, an appreciation of the wider field of computing, its aspects, and constituent technologies.
    \item To provide a social framework through which our members can support each other in technical matters (adhering to University policy regarding academic misconduct).
\end{itemize}

\section{Membership}

\textbf{Ordinary Membership.} Ordinary membership shall be open to all matricu-
lated students of the University of St Andrews.

\noindent \textbf{Associate Membership.} Associate membership shall be open to everybody
else.

\section{The Executive Committee}

The affairs of the society shall be managed by the Executive Committee, which
consists of:
\begin{itemize}
        \item \textbf{President}, who is responsible for:
            \begin{itemize}
                \item Chairing meetings of the committee/society
                \item Liaising with the School of Computer Science and other University organisations
                \item Organisation of elections within the society
                \item Liaising with the Student Union
            \end{itemize}
        \item \textbf{Secretary}, who is responsible for:
            \begin{itemize}
                \item Recording the minutes at committee/society meetings
                \item Managing the mailing list
                \item Social media publicity and marketing of society events
            \end{itemize}
        \item \textbf{Events Coordinator}, who is responsible for:
            \begin{itemize}
                \item Overall organisation of the society events
                \item Completion of supplementary risk assessments
            \end{itemize}
        \item \textbf{Sponsorship Coordinator}, who is responsible for:
            \begin{itemize}
                \item Liaising with current sponsors.
                \item Raising society sponsorship
                \item Outreach for new sponsorship opportunities.
            \end{itemize}
        \item \textbf{Society Treasurer}, who is responsible for:
            \begin{itemize}
                \item The financial administration of the society and its events.
                \item Management of society accounts.
            \end{itemize}
\end{itemize}

The above 5 members form the executive committee, which is responsible for:

\begin{itemize}
    \item Organisation of the society as a whole
    \item Dividing tasks between committee members and volunteers
\end{itemize}

\section{The General Committee}

In addition to the Executive Committee, there shall be a General Committee. Members of the General Committee shall be appointed by the Executive Committee with the intention to delegate organisational duties for events occurring during the academic year.

\subsection{Requirements}

All general committee members must be ordinary members of STACS.

\subsection{Appointment}

\begin{itemize}
    \item A prospective General Committee Member must be appointed by at least a 3/5 majority decision at an Executive Committee Meeting.
    \item All ordinary society members must be informed of the appointment by at most 7 days after the appointment.
\end{itemize}

General Committee membership lasts until either:

\begin{itemize}
    \item The member resigns;
    \item The member’s General Committee membership is revoked;
    \item or the end of the academic year.
\end{itemize}

\subsection{Revocation of Membership}

\begin{itemize}
    \item The Executive Committee reserves the right to revoke membership without giving reason.
    \item After revoking membership, all society members must be informed in the following 7 days.
\end{itemize}

\subsection{Resignation}

\begin{itemize}
    \item All members reserve the right to resign without giving reason.
    \item All executive committee members must be informed of the member’s decision.
    \item After resignation, all society members must be informed in the following 7 days.
\end{itemize}
\section{Competitive Programming Subcommittee}
    The group known as ``Competitive Programming St Andrews'', abbreviated to ``CPSTA'',  will exist as a subcommittee of STACS. They will operate semi-autonomously, and have their own committee. STACS will represent them in all matters relating the Student Association.
        \subsection{Aims}
        The aims of the subcommittee are to:
            \begin{itemize}
                \item Organise events related to improving algorithmic and problem solving skills.
                \item Organise the attendance of competitive programming events.
            \end{itemize}
        \subsection{Membership}
            All ordinary members of STACS will be eligible for membership of the subcommittee.
        \subsection{Executive Committee}
            The affairs of the subcommittee will be managed by an executive committee, which consists of five \textbf{Executive Committee Members}. They hold joint responsibility of:
            \begin{itemize}
                \item Organising Events
                \item Moderating social media outlets and discussion boards
                \item Social media publicity and marketing of society events
                \item Liaising with the STACS executive committee when necessary.
            \end{itemize}
            \subsubsection{Requirements}
                All Executive Committee members must be ordinary members of STACS.
            \subsubsection{Appointment}
                The Executive Committee will be appointed at the AGM, in line with standard Student's Association voting rules.
        \subsection{General Committee}
            In addition to the Executive Committee, there shall be a General Committee. Members of the General Committee shall be appointed by the Executive Committee with the intention to delegate organisational duties for events occurring during the academic year.
            \subsubsection{Requirements}
                All General Committee members must be ordinary members of STACS.
            \subsubsection{Appointment}
                \begin{itemize}
                    \item A prospective General Committee Member must be appointed by at least a 3/5 majority decision at an Executive Committee Meeting.
                    \item All subcommittee members must be informed of the appointment by at most 7 days after the appointment.
                \end{itemize}
            General Committee membership lasts until either:
            \begin{itemize}
                \item The member resigns;
                \item The member’s General Committee membership is revoked;
                \item or the end of the academic year.
            \end{itemize}
            \subsubsection{Revocation of Membership}
                \begin{itemize}
                    \item The Executive Committee reserves the right to revoke membership without giving reason.
                    \item After revoking membership, all subcommittee members must be informed in the following 7 days.
                \end{itemize}

            \subsubsection{Resignation}
                \begin{itemize}
                    \item All members reserve the right to resign without giving reason.
                    \item All executive committee members must be informed of the member’s decision.
                    \item After resignation, all subcommittee members must be informed in the following 7 days.
                \end{itemize}
\section{Subscription}

Society subscription shall be £3 for ordinary members, and £6 for associate
members.

\section{Standing Orders}

The committee shall follow Standing Orders for all affiliated societies as laid down by the Societies Committee of the Students’ Association.

\section{Finance}

The finances of the society will be dealt with following the guidelines laid down
by the Societies’ Committee of the Student’s Association. Two signatories of the Students’ Association shall appear on the Society’s bank mandate.

No Officer shall derive any financial profit or gain by reason of his officership, including favourable rates on goods or services, unless the same benefit is available to any and all members of the Society. In the event of a budget deficit, this will be paid off by an equal subscription from all members, unless the Societies’ Committee judges a negligence on the part of one or more office-bearers of the society when they as individuals will be liable. 

Membership fees will be in line with those set by the Societies’ Committee, and approved at the AGM.

\section{Outside Affiliation}

The society is entitled to form its own outside affiliations as laid down by the
Societies’ Committee of the Student’s Association. The society has no current
outside affiliations.

\section{Quorom}

The quorum for a committee meeting of the society will be at least 3/5 of all
elected committee members. All elected committee members will be given at
least 24 hours notice of any committee meeting.

The quorum for a General Meeting of the society will be at least 1/5 of
ordinary members. In the event of an inquorate meeting all members and the Societies’ Officer will be informed. If within 14 days there is no objection re-
ported, it will stand ratified.

In the presence of the Societies’ Officer, any meeting will be deemed to be
quorate.

\section{The Constitution}

A copy of this constitution will be available at every committee meeting and
general meeting of the society. Any proposed alteration must be passed by
the executive committee, and requires a 2/3 majority of members present at a
General Meeting. Any change must then be passed on for ratification by the
Societies’ Committee of the Student’s Association.

\section{Dissolution}

In the event of the club being dissolved any funds remaining in the society
account(s), and assets belonging to the society shall be donated to Computer
Aid International (Registered Charity Number 1069256).


\end{document}


